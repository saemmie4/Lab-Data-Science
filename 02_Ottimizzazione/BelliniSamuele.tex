\documentclass[a4paper]{article}
\usepackage{mathtools}
\usepackage{listings}
\usepackage{graphicx}
\usepackage{float}
\usepackage{amsmath}
\usepackage[font=it, width=0.9\linewidth]{caption}
\usepackage{bm}
\usepackage[hidelinks]{hyperref}
% \usepackage{placeins} %non fa andare le figure in una sezione nella sezione dopo%
% \usepackage[italian]{babel}
\newcounter{count_es}
% \captionsetup[figure]{name=Fig.}
\renewcommand{\figurename}{Fig.}
\renewcommand{\tablename}{Tab.}

\begin{document}
\title{Relazione di laboratorio \\Discesa del gradiente e regressione polinomiale}
\author{Samuele Bellini}
\date{8 Ottobre 2025}
\maketitle

\stepcounter{count_es}
\section*{Esercizio \arabic{count_es} - Generazione dei dati}
I dati da generare sono 200 coppie \((x_{i}, y_{i})\), con \(x_{i}\)
estratto da una distribuzione uniforme nell'intervallo [0,1] e \(y_{i} = \sin(2 \pi x_{i}) + \epsilon_{i}\), 
dove \(\epsilon_{i}\) rappresenta un rumore nella misura ed è estratto da una distribuzione uniforme nell'intervallo [-0.2,0.2].
Questi punti possono essere generati in Python tramite le linee di codice:
\begin{lstlisting}[language=Python]
    def RandPoint():
        x   = np.random.rand()
        eps = np.random.uniform(low=-0.2, high=0.2)
        return [x, np.sin(2*np.pi*x) + eps]
    N  = 200
    xy = np.array([RandPoint() for i in np.arange(N)]).T    
\end{lstlisting}
Si ottiene così il set di dati raffigurato in \figurename~\ref{fig:es_1}, assieme
alla curva senza il rumore dato da \(\epsilon_{i}\); dal grafico si vede bene come 
i dati seguono la curva distanziandosi al più di 0.2 unità da essa.
\begin{figure}[H]
    \centering
    \includegraphics[width=.95\linewidth]{Immagini/Es_1.png}
    \caption{Grafico dei dati generati con il rumore. In blu sono
    riportate le 200 coppie \((x_{i}, y_{i})\), in rosso la curva \(y = \sin(2 \pi x)\). \label{fig:es_1}}
\end{figure}
\end{document}